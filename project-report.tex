\documentclass[12pt,a4paper]{article}

% Essential packages
\usepackage[utf8]{inputenc}
\usepackage[margin=2.5cm]{geometry}
\usepackage{graphicx}
\usepackage{float}
\usepackage[hidelinks]{hyperref}
\usepackage{booktabs}
\usepackage{pgfplots}
\usepackage{appendix}

% Document title and author information
\title{\textbf{Data-Driven Analysis of Alumni Success: Analyzing Employment Outcomes in the Data Analytics Engineering Program}}
\author{James Baldo, Lingyun Dai}
\date{\today}

\begin{document}

\maketitle

% Abstract
\begin{abstract}
The effectiveness of Data Analytics Engineering (DAEN) program at George 
Mason University is critical as industry demand for skilled professionals 
continues to grow. This study presents analysis of the Data Analytics 
Engineering (DAEN) program at George Mason University's College of 
Computing and Engineering, examining post-graduation outcomes and 
program effectiveness through detailed alumni feedback. The 
research methodology employs a mixed-methods approach, combining 
qualitative interviews with quantitative analysis to evaluate 
various success metrics including employment trajectories, 
technical skill utilization, and program satisfaction. Data was 
collected through structured interviews with DAEN program alumni, 
focusing on key indicators such as job titles, career progression, 
technology stack adoption, and program-specific skill acquisition. 
Interview transcripts were processed using AWS transcription services 
and analyzed using natural language processing toolkit to extract 
meaningful patterns and insights. The processed data was then 
visualized using Tableau to identify trends in employment outcomes, 
skill utilization, and areas for program enhancement. Results from 
this analysis provide actionable insights for curriculum development 
and program improvements, while also offering valuable metrics for 
assessing the program's effectiveness in preparing graduates for 
industry demands. This research contributes to the broader 
understanding of Data Analytics Engineering (DAEN) program 
effectiveness.

\textbf{Keywords:} Data Analytics Engineering (DAEN), 
Alumni, Employment, Program, Data, Analysis
\end{abstract}

\newpage
\tableofcontents
\newpage

% 1. Introduction Section
\section{Introduction}
\subsection{Purpose}
The primary objective of this research is to evaluate the Data 
Analytics Engineering (DAEN) program's effectiveness through 
comprehensive alumni interviews. Through structured feedback 
obtained from recent program graduates, this study conducts a 
detailed analysis of employment outcomes and program performance 
metrics. These insights are instrumental in enabling the DAEN 
department to implement strategic enhancements to the program, 
encompassing curriculum development, technology integration, 
and key focus areas for student development and satisfactory.

\subsection{Readership}
The report is intended for the DAEN program academic and 
administrative personnel including leaderships, key stackholders, 
and development team. The findings and recommendations presented 
in this report will provide valuable insights for improving program 
effectiveness, curriculum design, and student outcomes.

\subsection{Doc Structure}
This report is organized into seven main sections. The Introduction 
establishes the research context and objectives. The Problem 
Statement formulates the core research questions and scope. 
The Data section outlines the interview-based data collection 
methodology and questionnaire design. The Analysis section 
details the data processing pipeline and techniques employed. 
The Visualization section presents the derived data visualizations 
and their interpretations. The Findings section synthesizes key 
insights obtained through visual analytics. Finally, the Next 
Steps and Lessons Learned section proposes future research 
directions and methodological improvements.

% 2. Problem Statement Section
\section{Problem Statement}
\subsection{Alumni Feedback}
[Detail the feedback received from alumni]

\subsection{Focus}
[Describe the main focus areas of the project]

\subsection{Problem}
[Clearly state the problem being addressed]

% 3. Data Section
\section{Data}
\subsection{Collection Process}
[Describe how data was collected through interviews]

\subsection{Questions}
[List and explain the interview questions]

\subsection{Data Process}
[Explain how the data was processed and summarized]

\subsection{Data Quality}
[Discuss the quality and reliability of the data]

% 4. Analysis Section
\section{Analysis}
[Present your detailed analysis]

% 5. Visualization Section
\section{Visualization}
% Include your graphs, charts, and visual representations
\begin{figure}[H]
    \centering
    % \includegraphics[width=0.8\textwidth]{your-image}
    \caption{Your caption here}
    \label{fig:your-label}
\end{figure}

% 6. Findings Section
\section{Findings}
[Present your key findings]

% 7. Next Steps/Lessons Learned Section
\section{Next Steps and Lessons Learned}
\subsection{Next Steps}
[Outline future recommendations]

\subsection{Lessons Learned}
[Discuss key takeaways and learning points]

% Appendices
\begin{appendices}
\section{Background}
[Additional background information]

\section{References}
\begin{thebibliography}{9}
    \bibitem{key1} Author, A. (Year). Title. Journal/Publisher.
\end{thebibliography}
\end{appendices}

\end{document}