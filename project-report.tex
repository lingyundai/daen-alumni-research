\documentclass[12pt,a4paper]{article}

% Essential packages
\usepackage[utf8]{inputenc}
\usepackage[margin=2.5cm]{geometry}
\usepackage{graphicx}
\usepackage{float}
\usepackage[hidelinks]{hyperref}
\usepackage{booktabs}
\usepackage{pgfplots}
\usepackage{appendix}

% Document title and author information
\title{\textbf{Enhancing DAEN Program Effectiveness: Analysis of Alumni Interview Feedback at George Mason University}}
\author{Lingyun Dai, James Baldo}
\date{\today}

\begin{document}

\maketitle

% Abstract
\begin{abstract}
The effectiveness of the DAEN program at George 
Mason University is critical as the department aim to provide industry-relevant education, focusing on tools, technologies, and skills that are directly applicable in professional roles. This study presents analysis of the DAEN program at the College of Computing and Engineering at George Mason University, examining industry demands and program effectiveness through comprehensive alumni feedback. The research is conducted through alumni interviews. The interviews are designed with structured qualitative and quantitative questions to evaluate various aspects of program alumni including academic profile, employment trajectories, technical skill utilization in jobs, program technical skill utilization, and program feedback. Data was collected through anonymous interview recordings in either audio or text format based on the interview platform used. Audio interview recordings were processed confidentially using AWS transcribe to generate text transcripts. The text transcripts were cleaned and extracted into columns in a CSV file using local natural language processing toolkit. The data was visualized using Tableau to identify patterns in employment outcomes, 
skill utilization, and areas for program enhancement. The data visualizations provide actionable insights for curriculum development 
and program improvements, while also offering valuable metrics for 
assessing the program's effectiveness in preparing graduates for 
industry demands. This research contributes to the broader 
understanding of DAEN program effectiveness.

\textbf{Keywords:} DAEN, Alumni, Employment, Program, Data,
 Analysis, Skill, Interview, Industry
\end{abstract}

\newpage
\tableofcontents
\newpage

\section*{List of Abbreviations}
\begin{tabular}{ll}
DAEN & Data Analytics Engineering\\
% Add other abbreviations used in your paper
\end{tabular}
\newpage

% 1. Introduction Section
\section{Introduction}
\subsection{Purpose}
The primary objective of this research is to evaluate the DAEN 
program's effectiveness through comprehensive alumni interviews. Through structured feedback 
obtained from program alumni, this study conducts a 
detailed analysis of employment outcomes and program performance 
metrics. These insights are instrumental in enabling the DAEN 
department to implement strategic enhancements to the program's curriculum development and technology integration.

\subsection{Readership}
The report is intended for the DAEN program academic and 
administrative personnel including leaderships, key stackholders, 
and development team. The findings and recommendations presented 
in this report will provide valuable insights for improving program 
effectiveness, curriculum design, and student outcomes.

\subsection{Doc Structure}
This report is organized into seven main sections. The Introduction 
establishes the research context and objectives. The Problem 
Statement formulates the core research questions and scope. 
The Data section outlines the interview-based data collection 
methodology and questionnaire design. The Analysis section 
details the data processing pipeline and techniques employed. 
The Visualization section presents the derived data visualizations 
and their interpretations. The Findings section synthesizes key 
insights obtained through visual analytics. Finally, the Next 
Steps and Lessons Learned section proposes future research 
directions and methodological improvements.

% 2. Problem Statement Section
\section{Problem Statement}
\subsection{Alumni Feedback}
Analysis of feedback from DAEN program alumni between 2019-2023 reveals significant insights into program effectiveness. Across 13 alumni respondents, the average program satisfaction rating is 3.7 out of 5, indicating general satisfaction but with room for improvement. Alumni consistently emphasized the value of practical skills acquired through courses like database systems, statistical analysis, and machine learning. However, they also highlighted gaps between academic preparation and industry requirements, particularly in areas of cloud computing and hands-on projects.

\subsection{Focus}
This research examines the DAEN program's effectiveness through analysis of alumni career trajectories and employment outcomes. Our data shows alumni have secured positions across various roles including Data Scientists, Data Engineers, Data Analysts, and Machine Learning Engineers at prominent organizations such as Fannie Mae, Capital One, and Activision Blizzard. The study specifically focuses on evaluating how well the program's curriculum aligns with industry technology requirements, with particular attention to programming languages, cloud platforms, and analytical tools commonly used in professional settings.

\subsection{Problem}
The DAEN program faces key challenges in obtaining comprehensive alumni data to evaluate the program effectiveness. After the alumni feedback was gathered, some key problems with the program was discovered. First, while the program provides strong theoretical foundations, alumni feedback indicates a need for increased hands-on experience with industry tools and technologies. The data reveals that while 85\% of alumni work with cloud platforms (AWS, Azure, GCP) in their current roles, many received limited exposure to these technologies during their coursework. Second, there is a significant demand for enhanced project-based learning, with multiple alumni suggesting earlier integration of capstone projects and more extensive project planning experience. Third, the technology stack taught in the program requires continuous updating - while many courses focus on R programming, industry positions predominantly require Python proficiency, as evidenced by 77\% of alumni reporting Python as a primary tool in their current roles. Additionally, emerging areas such as AI ethics, DataOps, and machine learning operations need stronger representation in the curriculum to better prepare students for current industry demands. These challenges must be addressed to maintain the program's effectiveness in preparing graduates for successful careers.

% 3. Data Section
\section{Data}
\subsection{Collection Process}
The data collection process was conducted through structured interviews with DAEN program alumni. Alumni were identified and contacted through LinkedIn for interview scheduling. To ensure consistency and comprehensive data gathering, each interview followed a standardized set of questions covering academic background, career progression, technology and tools used, and program feedback. Interviews were conducted remotely and recorded with permission while maintaining participant anonymity throughout the process.

\subsection{Questions}
The interviews were structured with the following 12 questions in the same order for each participating alumni:

1. What year and spring/fall did you graduate from the Data Analytics Engineering (DAEN) program? 

This question aims to establish the alumni's graduation year and semester to track the program's impact over time.

2. Did you receive an M.S. or Certificate?  

3. What was the title of your first job, the name of the company and the general responsibilities?

4. What technologies/tools did you use for this job title? 

5. What is your current job title, the name of the company and the general responsibilites?  

6. How many jobs have you had since you graduated? 

7. List the most used technologies/tools in your career. (E.g. Programming language, framework, cloud, ML) 

8. What knowledge and skills that you acquired in the DAEN program have been the most valuable to your career? Can you specify the concepts/methodologies/technologies that were most valuable? 

9. If DAEN program provided these specific courses, topics, or training, I would have been more prepared in my career… 

10. How well did the DAEN courses prepare you for your career? (Scale: 1 – Not well at all, 5 – Very good) 

11. Have you completed any courses/certifications since you graduated from the DAEN program? 

\subsection{Data Process}
[Explain how the data was processed and summarized]

\subsection{Data Quality}
[Discuss the quality and reliability of the data]

% 4. Analysis Section
\section{Analysis}
[Present your detailed analysis]

% 5. Visualization Section
\section{Visualization}
% Include your graphs, charts, and visual representations
\begin{figure}[H]
    \centering
    % \includegraphics[width=0.8\textwidth]{your-image}
    \caption{Your caption here}
    \label{fig:your-label}
\end{figure}

% 6. Findings Section
\section{Findings}
[Present your key findings]

% 7. Next Steps/Lessons Learned Section
\section{Next Steps and Lessons Learned}
\subsection{Next Steps}
[Outline future recommendations]

\subsection{Lessons Learned}
[Discuss key takeaways and learning points]

% Appendices
\begin{appendices}
\section{Background}
[Additional background information]

\section{References}
\begin{thebibliography}{9}
    \bibitem{key1} Author, A. (Year). Title. Journal/Publisher.
\end{thebibliography}
\end{appendices}

\end{document}